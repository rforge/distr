\documentclass{article}
\usepackage[ae,hyper]{Rd}
\begin{document}
\HeaderA{distrTeach-package}{distrTeach -- Teaching Extensions of package distr}{distrTeach.Rdash.package}
\aliasA{distrTeach}{distrTeach-package}{distrTeach}
\keyword{package}{distrTeach-package}
\begin{Description}\relax
\pkg{distrTeach} provides some illustrations based on package \pkg{distr}
for teaching Stochastics / Statistics in secondary school; so far the following
has been implemented
\Itemize{
\item \code{illustrateLLT}: function for the generation of LLN - visualizations
\item  \code{illustrateCLT}: function for the generation of CLT - visualizations
\item  \code{plotCLT}: Generic function for the plotting of CLT-approximations
}
as well as a Tcl/Tk based demo for\code{illustrateCLT}
\end{Description}
\begin{Details}\relax
\Tabular{ll}{
Package: & distrTeach\\
Version: & 2.0 \\
Date: & 2008-01-15 \\
Depends: & R(>= 2.2.0), methods, distr(>= 1.8), evd, startupmsg\\
Suggests: & tcltk\\
SaveImage: & no\\
LazyLoad: & yes\\
License: & GPL (version 2 or later)\\
URL: & http://www.uni-bayreuth.de/departments/math/org/mathe7/DISTR/\\
}
\end{Details}
\begin{Section}{Classes}
\begin{alltt}

Teaching Classes

\end{alltt}
\end{Section}
\begin{Section}{Functions}
\begin{alltt}

\end{alltt}
\end{Section}
\begin{Section}{Generating Functions}
\begin{alltt}


\end{alltt}
\end{Section}
\begin{Section}{Methods}
\begin{alltt}

illustration:
illustrateLLT           function for the generation of LLN - visualizations
illustrateCLT           function for the generation of CLT - visualizations
plotCLT                 Generic function for the plotting of CLT-approximations

\end{alltt}
\end{Section}
\begin{Section}{Demos}
Demos are available --- see \code{demo(package="distrTeach")}.
\end{Section}
\begin{Section}{Start-up-Banner}
You may suppress the start-up banner/message completely by setting 
\code{options("StartupBanner"="off")} somewhere before loading this package by 
\code{library} or \code{require} in your R-code / R-session.

If option \code{"StartupBanner"} is not defined (default) or setting    
\code{options("StartupBanner"=NULL)} or 
\code{options("StartupBanner"="complete")} the complete start-up banner is 
displayed.

For any other value of option \code{"StartupBanner"} (i.e., not in 
\code{c(NULL,"off","complete")}) only the version information is displayed.

The same can be achieved by wrapping the \code{library} or \code{require}  call 
into either \code{suppressStartupMessages()} or 
\code{onlytypeStartupMessages(.,atypes="version")}.  

As for general \code{packageStartupMessage}'s, you may also suppress all
the start-up banner by wrapping the \code{library} or \code{require} 
call into \code{suppressPackageStartupMessages()} from 
\pkg{startupmsg}-version 0.5 on.
\end{Section}
\begin{Author}\relax
Matthias Kohl \email{Matthias.Kohl@stamats.de} and \\
Peter Ruckdeschel \email{Peter.Ruckdeschel@uni-bayreuth.de},\\ 
Eleonara Feist \email{eleonoragerber@gmx.de}, and, \\
Anja Hueller \email{anja_h86@web.de}\\  

\emph{Maintainer:}  Peter Ruckdeschel \email{Peter.Ruckdeschel@uni-bayreuth.de}
\end{Author}
\begin{References}\relax
P. Ruckdeschel, M. Kohl, T. Stabla, F. Camphausen (2006):
S4 Classes for Distributions, {\em R News}, {\em 6}(2), 2-6. 
\url{http://CRAN.R-project.org/doc/Rnews/Rnews_2006-2.pdf}


a vignette for packages \pkg{distr}, \pkg{distrSim}, \pkg{distrTEst}, 
and \pkg{distrTeach} is included into the mere documentation package \pkg{distrDoc} 
and may be called by \code{require("distrDoc");vignette("distr")}

a homepage to this package is available under\\
\url{http://distr.r-forge.r-project.org/} and the pages ...

M. Kohl (2005): \emph{Numerical Contributions to the Asymptotic 
Theory of Robustness.} PhD Thesis. Bayreuth. Available as 
\url{http://www.stamats.de/ThesisMKohl.pdf}
\end{References}
\begin{SeeAlso}\relax
\code{\LinkA{distr}{distr}} 
\code{\LinkA{distrEx}{distrEx}}
\end{SeeAlso}

\HeaderA{illustrateCLT}{Functions for Illustrating the CLT}{illustrateCLT}
\methaliasA{illustrateCLT.tcl}{illustrateCLT}{illustrateCLT.tcl}
\keyword{distribution}{illustrateCLT}
\keyword{methods}{illustrateCLT}
\keyword{dynamic}{illustrateCLT}
\begin{Description}\relax
Functions for generating a sequence of plots of
the density and cdf of the consecutive standardized and centered sums of iid 
r.v. distributed according to a prescribed discrete or absolutely continuous 
distribution compared to the standard normal --- uses the generic function 
\code{plotCLT}.
\end{Description}
\begin{Usage}
\begin{verbatim}illustrateCLT(Distr, len, sleep = 0)
illustrateCLT.tcl(Distr, k, Distrname)
\end{verbatim}
\end{Usage}
\begin{Arguments}
\begin{ldescription}
\item[\code{Distr}] object of class \code{"AbscontDistribution"}, 
\code{"LatticeDistribution"} or \code{"DiscreteDistribution"}: distribution of 
the summands
\item[\code{len}] integer: up to which number of summands plots are generated
\item[\code{k}] integer: number of summands for which a plot is to be generated
\item[\code{Distrname}] character: name of the summand distribution to be used as 
title in the plot
\item[\code{sleep}] numeric: pause in seconds between subsequent plots 
\end{ldescription}
\end{Arguments}
\begin{Details}\relax
\code{illustrateCLT} generates a sequence of plots, while 
\code{illustrateCLT.tcl} may be used with Tcl/Tk-widgets as in demo 
\code{illustCLT\_tcl.R}.
\end{Details}
\begin{Value}
void
\end{Value}
\begin{Author}\relax
Matthias Kohl \email{Matthias.Kohl@stamats.de}\\
Peter Ruckdeschel \email{Peter.Ruckdeschel@uni-bayreuth.de}
\end{Author}
\begin{SeeAlso}\relax
\code{\LinkA{plotCLT}{plotCLT}}
\end{SeeAlso}
\begin{Examples}
\begin{ExampleCode}
distroptions("DefaultNrFFTGridPointsExponent" = 13)
illustrateCLT(Distr = Unif(), len = 20)
distroptions("DefaultNrFFTGridPointsExponent" = 12)
illustrateCLT(Distr = Pois(lambda = 2), len = 20)
distroptions("DefaultNrFFTGridPointsExponent" = 13)
illustrateCLT(Distr = Pois(lambda = 2)+Unif(), len = 20)
illustrateCLT.tcl(Distr = Unif(), k = 4, "Unif()")
\end{ExampleCode}
\end{Examples}

\HeaderA{illustrateLLN}{Functions for Illustrating the LLN}{illustrateLLN}
\keyword{distribution}{illustrateLLN}
\keyword{methods}{illustrateLLN}
\keyword{dynamic}{illustrateLLN}
\begin{Description}\relax
Functions for generating a sequence of plots of
randomly generated replicates of 
\eqn{\bar X_n= \frac{1}{n} \sum_{i=1}^n X_i}{arithmetic means}
for sums of iid r.v. distributed according to a prescribed discrete or 
absolutely continuous distribution. A line for the expectation and CLT based
(pointwise) 95\%-confidence bands are also plotted and the empirical coverage
of this band by the replicated plotted so far is indicated.
\end{Description}
\begin{Usage}
\begin{verbatim}illustrateLLN(Distr = Norm(),n = c(1,3,5,10,25,50,100,500,1000,10000), 
            m = 50, step = 1, sleep = 0, withConf = TRUE, 
            withCover = (length(n)<=12), withEline = TRUE, withLegend = TRUE,
            CLTorCheb = "CLT",  coverage = 0.95, ...,  col.Eline = "blue", 
            lwd.Eline = par("lwd"), lty.Eline = par("lty"), col.Conf = "red", 
            lwd.Conf = par("lwd"), lty.Conf = 2, cex.Cover = 0.7, 
            cex.legend = 0.8)
\end{verbatim}
\end{Usage}
\begin{Arguments}
\begin{ldescription}
\item[\code{Distr}] object of class \code{"UnivariateDistribution"}: 
distribution of the summands
\item[\code{n}] vector of integers: sample sizes to be considered
\item[\code{m}] integer: (total) number of replicates to be plotted subsequently
\item[\code{step}] integer: number of replicates to be drawn at once
\item[\code{sleep}] numeric: pause in seconds between subsequent plots 
\item[\code{withEline}] logical: shall a line for the limiting expectation 
(in case of class \code{Cauchy} instead: median)
be drawn?
\item[\code{withConf}] logical: shall (CLT-based) confidence bands be plotted?
\item[\code{withCover}] logical: shall empirical coverage of (CLT-based) confidence 
bands be printed?
\item[\code{withLegend}] logical: shall a legend be included?
\item[\code{CLTorCheb}] character: type of confidence interval ---"CLT" or
"Chebyshev"; partial matching is used; if this fails
"CLT" is used.
\item[\code{coverage}] numerical: nominal coverage of the confidence bands 
---to be in (0,1)
\item[\code{col.Eline}] character or integer code; color for confidence bands
\item[\code{lwd.Eline}] integer code (see \code{\LinkA{par}{par}}); 
line width of the confidence bands
\item[\code{lty.Eline}] integer code (see \code{\LinkA{par}{par}}); 
line type of the confidence bands
\item[\code{col.Conf}] character or integer code; color for confidence bands
\item[\code{lwd.Conf}] integer code (see \code{\LinkA{par}{par}}); 
line width of the confidence bands
\item[\code{lty.Conf}] integer code (see \code{\LinkA{par}{par}}); 
line type of the confidence bands
\item[\code{cex.Cover}] magnification w.r.t. the current setting of \code{cex} 
to be used for empirical coverages; as in 
\code{\LinkA{par}{par}}
\item[\code{cex.legend}] magnification w.r.t. the current setting of \code{cex} 
to be used for the legend as in 
\code{\LinkA{par}{par}}
\item[\code{...}] further arguments to be passed to \code{matplot},
\code{matlines}, \code{abline}
\end{ldescription}
\end{Arguments}
\begin{Details}\relax
\code{illustrateLLN} generates a sequence of plots.
Any parameters of \code{plot.default} may be passed on to this particular
\code{plot} method. 

There are default \code{main} titles as well as \code{xlab} and \code{ylab}
annotations.

In all title arguments, the following patterns are substituted:
\Itemize{
\item[\code{"\%C"}] class of argument \code{x}
\item[\code{"\%P"}] parameters of \code{x} in form of a comma-separated list of
<value>'s coerced to character
\item[\code{"\%Q"}] parameters of \code{x} in form of a comma-separated list of
<value>'s coerced to character and in parenthesis --- unless
empty; then ""
\item[\code{"\%N"}] parameters of \code{x} in form of a comma-separated list
<name> = <value> coerced to character
\item[\code{"\%A"}] deparsed argument \code{x}
\item[\code{"\%D"}] time/date-string when the plot was generated
\item[\code{"\%X"}] the expression
\eqn{\bar X_n=\sum_{i=1}^n X_i/n}{\code{bar(X)[n]==\textasciitilde{}\textasciitilde{}sum(X[i],i==1,n)/n}}
}

If not explicitly set, \code{col.Eline}, \code{col.Conf} are set 
to \code{col} if this arg is given and else to their default values as given
above. Similarly for \code{cex}, \code{lwd} and \code{lty}.
\end{Details}
\begin{Value}
void
\end{Value}
\begin{Author}\relax
Peter Ruckdeschel \email{Peter.Ruckdeschel@uni-bayreuth.de}
\end{Author}
\begin{Examples}
\begin{ExampleCode}
illustrateLLN(Distr = Unif())
illustrateLLN(Distr = Pois(lambda = 2))
illustrateLLN(Distr = Pois(lambda = 2)+Unif())
illustrateLLN(Td(3), m = 50, col.Eline = "green", lwd = 2, cex = 0.6, main = 
 "My LLN %C%Q", sub = "generated %D")
illustrateLLN(Td(3), m = 50, CLTorCheb = "Chebyshev") 
illustrateLLN(Td(3), m = 50, CLTorCheb = "Chebyshev", coverage = 0.75) 
\end{ExampleCode}
\end{Examples}

\HeaderA{plotCLT}{Generic Plot Function for Illustrating the CLT}{plotCLT}
\aliasA{plotCLT,AbscontDistribution-method}{plotCLT}{plotCLT,AbscontDistribution.Rdash.method}
\aliasA{plotCLT,DiscreteDistribution-method}{plotCLT}{plotCLT,DiscreteDistribution.Rdash.method}
\aliasA{plotCLT-methods}{plotCLT}{plotCLT.Rdash.methods}
\keyword{internal}{plotCLT}
\keyword{methods}{plotCLT}
\keyword{hplot}{plotCLT}
\keyword{distribution}{plotCLT}
\begin{Description}\relax
Generic 'plot' function for generating the plots of 'illustrateCLT'.
\end{Description}
\begin{Usage}
\begin{verbatim}plotCLT(Tn, ...)
## S4 method for signature 'AbscontDistribution':
plotCLT(Tn, k, summands="")
## S4 method for signature 'DiscreteDistribution':
plotCLT(Tn, k, summands="")
\end{verbatim}
\end{Usage}
\begin{Arguments}
\begin{ldescription}
\item[\code{Tn}] object of class \code{"AbscontDistribution"} or class
\code{"DiscreteDistribution"}: distribution of the summands
\item[\code{k}] integer: number off summands to be plotted as graphics title
\item[\code{summands}] character: name of the summands
\item[\code{...}] addtional arguments for methods not yet implemented 
\end{ldescription}
\end{Arguments}
\begin{Value}
void
\end{Value}
\begin{Author}\relax
Peter Ruckdeschel \email{Peter.Ruckdeschel@uni-bayreuth.de}\\
Matthias Kohl \email{Matthias.Kohl@stamats.de}
\end{Author}
\begin{SeeAlso}\relax
\code{\LinkA{illustrateCLT}{illustrateCLT}}
\end{SeeAlso}
\begin{Examples}
\begin{ExampleCode}
illustrateCLT(Distr = Unif(), len = 20)
\end{ExampleCode}
\end{Examples}

\end{document}
