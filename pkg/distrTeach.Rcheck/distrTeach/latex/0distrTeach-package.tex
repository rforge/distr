\HeaderA{distrTeach-package}{distrTeach -- Teaching Extensions of package distr}{distrTeach.Rdash.package}
\aliasA{distrTeach}{distrTeach-package}{distrTeach}
\keyword{package}{distrTeach-package}
\begin{Description}\relax
\pkg{distrTeach} provides some illustrations based on package \pkg{distr}
for teaching Stochastics / Statistics in secondary school; so far the following
has been implemented
\Itemize{
\item \code{illustrateLLT}: function for the generation of LLN - visualizations
\item  \code{illustrateCLT}: function for the generation of CLT - visualizations
\item  \code{plotCLT}: Generic function for the plotting of CLT-approximations
}
as well as a Tcl/Tk based demo for\code{illustrateCLT}
\end{Description}
\begin{Details}\relax
\Tabular{ll}{
Package: & distrTeach\\
Version: & 2.0 \\
Date: & 2008-01-15 \\
Depends: & R(>= 2.2.0), methods, distr(>= 1.8), evd, startupmsg\\
Suggests: & tcltk\\
SaveImage: & no\\
LazyLoad: & yes\\
License: & GPL (version 2 or later)\\
URL: & http://www.uni-bayreuth.de/departments/math/org/mathe7/DISTR/\\
}
\end{Details}
\begin{Section}{Classes}
\begin{alltt}

Teaching Classes

\end{alltt}
\end{Section}
\begin{Section}{Functions}
\begin{alltt}

\end{alltt}
\end{Section}
\begin{Section}{Generating Functions}
\begin{alltt}


\end{alltt}
\end{Section}
\begin{Section}{Methods}
\begin{alltt}

illustration:
illustrateLLT           function for the generation of LLN - visualizations
illustrateCLT           function for the generation of CLT - visualizations
plotCLT                 Generic function for the plotting of CLT-approximations

\end{alltt}
\end{Section}
\begin{Section}{Demos}
Demos are available --- see \code{demo(package="distrTeach")}.
\end{Section}
\begin{Section}{Start-up-Banner}
You may suppress the start-up banner/message completely by setting 
\code{options("StartupBanner"="off")} somewhere before loading this package by 
\code{library} or \code{require} in your R-code / R-session.

If option \code{"StartupBanner"} is not defined (default) or setting    
\code{options("StartupBanner"=NULL)} or 
\code{options("StartupBanner"="complete")} the complete start-up banner is 
displayed.

For any other value of option \code{"StartupBanner"} (i.e., not in 
\code{c(NULL,"off","complete")}) only the version information is displayed.

The same can be achieved by wrapping the \code{library} or \code{require}  call 
into either \code{suppressStartupMessages()} or 
\code{onlytypeStartupMessages(.,atypes="version")}.  

As for general \code{packageStartupMessage}'s, you may also suppress all
the start-up banner by wrapping the \code{library} or \code{require} 
call into \code{suppressPackageStartupMessages()} from 
\pkg{startupmsg}-version 0.5 on.
\end{Section}
\begin{Author}\relax
Matthias Kohl \email{Matthias.Kohl@stamats.de} and \\
Peter Ruckdeschel \email{Peter.Ruckdeschel@uni-bayreuth.de},\\ 
Eleonara Feist \email{eleonoragerber@gmx.de}, and, \\
Anja Hueller \email{anja_h86@web.de}\\  

\emph{Maintainer:}  Peter Ruckdeschel \email{Peter.Ruckdeschel@uni-bayreuth.de}
\end{Author}
\begin{References}\relax
P. Ruckdeschel, M. Kohl, T. Stabla, F. Camphausen (2006):
S4 Classes for Distributions, {\em R News}, {\em 6}(2), 2-6. 
\url{http://CRAN.R-project.org/doc/Rnews/Rnews_2006-2.pdf}


a vignette for packages \pkg{distr}, \pkg{distrSim}, \pkg{distrTEst}, 
and \pkg{distrTeach} is included into the mere documentation package \pkg{distrDoc} 
and may be called by \code{require("distrDoc");vignette("distr")}

a homepage to this package is available under\\
\url{http://distr.r-forge.r-project.org/} and the pages ...

M. Kohl (2005): \emph{Numerical Contributions to the Asymptotic 
Theory of Robustness.} PhD Thesis. Bayreuth. Available as 
\url{http://www.stamats.de/ThesisMKohl.pdf}
\end{References}
\begin{SeeAlso}\relax
\code{\LinkA{distr}{distr}} 
\code{\LinkA{distrEx}{distrEx}}
\end{SeeAlso}

