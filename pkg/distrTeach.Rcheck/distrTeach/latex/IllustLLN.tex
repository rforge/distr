\HeaderA{illustrateLLN}{Functions for Illustrating the LLN}{illustrateLLN}
\keyword{distribution}{illustrateLLN}
\keyword{methods}{illustrateLLN}
\keyword{dynamic}{illustrateLLN}
\begin{Description}\relax
Functions for generating a sequence of plots of
randomly generated replicates of 
\eqn{\bar X_n= \frac{1}{n} \sum_{i=1}^n X_i}{arithmetic means}
for sums of iid r.v. distributed according to a prescribed discrete or 
absolutely continuous distribution. A line for the expectation and CLT based
(pointwise) 95\%-confidence bands are also plotted and the empirical coverage
of this band by the replicated plotted so far is indicated.
\end{Description}
\begin{Usage}
\begin{verbatim}illustrateLLN(Distr = Norm(),n = c(1,3,5,10,25,50,100,500,1000,10000), 
            m = 50, step = 1, sleep = 0, withConf = TRUE, 
            withCover = (length(n)<=12), withEline = TRUE, withLegend = TRUE,
            CLTorCheb = "CLT",  coverage = 0.95, ...,  col.Eline = "blue", 
            lwd.Eline = par("lwd"), lty.Eline = par("lty"), col.Conf = "red", 
            lwd.Conf = par("lwd"), lty.Conf = 2, cex.Cover = 0.7, 
            cex.legend = 0.8)
\end{verbatim}
\end{Usage}
\begin{Arguments}
\begin{ldescription}
\item[\code{Distr}] object of class \code{"UnivariateDistribution"}: 
distribution of the summands
\item[\code{n}] vector of integers: sample sizes to be considered
\item[\code{m}] integer: (total) number of replicates to be plotted subsequently
\item[\code{step}] integer: number of replicates to be drawn at once
\item[\code{sleep}] numeric: pause in seconds between subsequent plots 
\item[\code{withEline}] logical: shall a line for the limiting expectation 
(in case of class \code{Cauchy} instead: median)
be drawn?
\item[\code{withConf}] logical: shall (CLT-based) confidence bands be plotted?
\item[\code{withCover}] logical: shall empirical coverage of (CLT-based) confidence 
bands be printed?
\item[\code{withLegend}] logical: shall a legend be included?
\item[\code{CLTorCheb}] character: type of confidence interval ---"CLT" or
"Chebyshev"; partial matching is used; if this fails
"CLT" is used.
\item[\code{coverage}] numerical: nominal coverage of the confidence bands 
---to be in (0,1)
\item[\code{col.Eline}] character or integer code; color for confidence bands
\item[\code{lwd.Eline}] integer code (see \code{\LinkA{par}{par}}); 
line width of the confidence bands
\item[\code{lty.Eline}] integer code (see \code{\LinkA{par}{par}}); 
line type of the confidence bands
\item[\code{col.Conf}] character or integer code; color for confidence bands
\item[\code{lwd.Conf}] integer code (see \code{\LinkA{par}{par}}); 
line width of the confidence bands
\item[\code{lty.Conf}] integer code (see \code{\LinkA{par}{par}}); 
line type of the confidence bands
\item[\code{cex.Cover}] magnification w.r.t. the current setting of \code{cex} 
to be used for empirical coverages; as in 
\code{\LinkA{par}{par}}
\item[\code{cex.legend}] magnification w.r.t. the current setting of \code{cex} 
to be used for the legend as in 
\code{\LinkA{par}{par}}
\item[\code{...}] further arguments to be passed to \code{matplot},
\code{matlines}, \code{abline}
\end{ldescription}
\end{Arguments}
\begin{Details}\relax
\code{illustrateLLN} generates a sequence of plots.
Any parameters of \code{plot.default} may be passed on to this particular
\code{plot} method. 

There are default \code{main} titles as well as \code{xlab} and \code{ylab}
annotations.

In all title arguments, the following patterns are substituted:
\Itemize{
\item[\code{"\%C"}] class of argument \code{x}
\item[\code{"\%P"}] parameters of \code{x} in form of a comma-separated list of
<value>'s coerced to character
\item[\code{"\%Q"}] parameters of \code{x} in form of a comma-separated list of
<value>'s coerced to character and in parenthesis --- unless
empty; then ""
\item[\code{"\%N"}] parameters of \code{x} in form of a comma-separated list
<name> = <value> coerced to character
\item[\code{"\%A"}] deparsed argument \code{x}
\item[\code{"\%D"}] time/date-string when the plot was generated
\item[\code{"\%X"}] the expression
\eqn{\bar X_n=\sum_{i=1}^n X_i/n}{\code{bar(X)[n]==\textasciitilde{}\textasciitilde{}sum(X[i],i==1,n)/n}}
}

If not explicitly set, \code{col.Eline}, \code{col.Conf} are set 
to \code{col} if this arg is given and else to their default values as given
above. Similarly for \code{cex}, \code{lwd} and \code{lty}.
\end{Details}
\begin{Value}
void
\end{Value}
\begin{Author}\relax
Peter Ruckdeschel \email{Peter.Ruckdeschel@uni-bayreuth.de}
\end{Author}
\begin{Examples}
\begin{ExampleCode}
illustrateLLN(Distr = Unif())
illustrateLLN(Distr = Pois(lambda = 2))
illustrateLLN(Distr = Pois(lambda = 2)+Unif())
illustrateLLN(Td(3), m = 50, col.Eline = "green", lwd = 2, cex = 0.6, main = 
 "My LLN %C%Q", sub = "generated %D")
illustrateLLN(Td(3), m = 50, CLTorCheb = "Chebyshev") 
illustrateLLN(Td(3), m = 50, CLTorCheb = "Chebyshev", coverage = 0.75) 
\end{ExampleCode}
\end{Examples}

