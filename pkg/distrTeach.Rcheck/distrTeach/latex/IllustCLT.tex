\HeaderA{illustrateCLT}{Functions for Illustrating the CLT}{illustrateCLT}
\methaliasA{illustrateCLT.tcl}{illustrateCLT}{illustrateCLT.tcl}
\keyword{distribution}{illustrateCLT}
\keyword{methods}{illustrateCLT}
\keyword{dynamic}{illustrateCLT}
\begin{Description}\relax
Functions for generating a sequence of plots of
the density and cdf of the consecutive standardized and centered sums of iid 
r.v. distributed according to a prescribed discrete or absolutely continuous 
distribution compared to the standard normal --- uses the generic function 
\code{plotCLT}.
\end{Description}
\begin{Usage}
\begin{verbatim}illustrateCLT(Distr, len, sleep = 0)
illustrateCLT.tcl(Distr, k, Distrname)
\end{verbatim}
\end{Usage}
\begin{Arguments}
\begin{ldescription}
\item[\code{Distr}] object of class \code{"AbscontDistribution"}, 
\code{"LatticeDistribution"} or \code{"DiscreteDistribution"}: distribution of 
the summands
\item[\code{len}] integer: up to which number of summands plots are generated
\item[\code{k}] integer: number of summands for which a plot is to be generated
\item[\code{Distrname}] character: name of the summand distribution to be used as 
title in the plot
\item[\code{sleep}] numeric: pause in seconds between subsequent plots 
\end{ldescription}
\end{Arguments}
\begin{Details}\relax
\code{illustrateCLT} generates a sequence of plots, while 
\code{illustrateCLT.tcl} may be used with Tcl/Tk-widgets as in demo 
\code{illustCLT\_tcl.R}.
\end{Details}
\begin{Value}
void
\end{Value}
\begin{Author}\relax
Matthias Kohl \email{Matthias.Kohl@stamats.de}\\
Peter Ruckdeschel \email{Peter.Ruckdeschel@uni-bayreuth.de}
\end{Author}
\begin{SeeAlso}\relax
\code{\LinkA{plotCLT}{plotCLT}}
\end{SeeAlso}
\begin{Examples}
\begin{ExampleCode}
distroptions("DefaultNrFFTGridPointsExponent" = 13)
illustrateCLT(Distr = Unif(), len = 20)
distroptions("DefaultNrFFTGridPointsExponent" = 12)
illustrateCLT(Distr = Pois(lambda = 2), len = 20)
distroptions("DefaultNrFFTGridPointsExponent" = 13)
illustrateCLT(Distr = Pois(lambda = 2)+Unif(), len = 20)
illustrateCLT.tcl(Distr = Unif(), k = 4, "Unif()")
\end{ExampleCode}
\end{Examples}

