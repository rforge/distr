\HeaderA{plotCLT}{Generic Plot Function for Illustrating the CLT}{plotCLT}
\aliasA{plotCLT,AbscontDistribution-method}{plotCLT}{plotCLT,AbscontDistribution.Rdash.method}
\aliasA{plotCLT,DiscreteDistribution-method}{plotCLT}{plotCLT,DiscreteDistribution.Rdash.method}
\aliasA{plotCLT-methods}{plotCLT}{plotCLT.Rdash.methods}
\keyword{internal}{plotCLT}
\keyword{methods}{plotCLT}
\keyword{hplot}{plotCLT}
\keyword{distribution}{plotCLT}
\begin{Description}\relax
Generic 'plot' function for generating the plots of 'illustrateCLT'.
\end{Description}
\begin{Usage}
\begin{verbatim}plotCLT(Tn, ...)
## S4 method for signature 'AbscontDistribution':
plotCLT(Tn, k, summands="")
## S4 method for signature 'DiscreteDistribution':
plotCLT(Tn, k, summands="")
\end{verbatim}
\end{Usage}
\begin{Arguments}
\begin{ldescription}
\item[\code{Tn}] object of class \code{"AbscontDistribution"} or class
\code{"DiscreteDistribution"}: distribution of the summands
\item[\code{k}] integer: number off summands to be plotted as graphics title
\item[\code{summands}] character: name of the summands
\item[\code{...}] addtional arguments for methods not yet implemented 
\end{ldescription}
\end{Arguments}
\begin{Value}
void
\end{Value}
\begin{Author}\relax
Peter Ruckdeschel \email{Peter.Ruckdeschel@uni-bayreuth.de}\\
Matthias Kohl \email{Matthias.Kohl@stamats.de}
\end{Author}
\begin{SeeAlso}\relax
\code{\LinkA{illustrateCLT}{illustrateCLT}}
\end{SeeAlso}
\begin{Examples}
\begin{ExampleCode}
illustrateCLT(Distr = Unif(), len = 20)
\end{ExampleCode}
\end{Examples}

